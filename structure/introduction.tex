\chapterimage{head2.png} % Chapter heading image

\chapter{Introduction}

\section{Motivation}\index{Motivation}
When I applied for the summer research internship, the title of the project was \emph{The many colours of nearby galaxies} an the description was
\begin{quote}
The different populations of stars in a galaxy carry the record of its past star formation history, and also affect its future. The project involves  analyzing Hubble Space Telescopes images of nearby galaxies of different types. By measuring the brightness and colours of millions of stars, we can understand the ages and compositions of the stars, and learn how the galaxy formed stars in the past. The radiation emitted by stars affects the gas in a galaxy, and thus how it will form stars in the future. We will use multi-colour images of galaxies to gain new insights into both their past and future.
\end{quote}

So, as an engineer without any astrophysics background I thougth I would be doing image processing applied to astronomy and I ended up doing so much more, but hey! You never know what you will end up doing.

Before comming to Canada, Pauline and I exchanged some emails where she shared me some interesting papers, webpages and an astronomy online course which later I did take, mainly the information was about a general introduction to astronomy and how astronomy images are, yes, astronomy images are completely different as any other \emph{normal} images, they are made of purely science data and every image has valuable knowledge you can learn from, and hey you will forget soon about pixels and start talking about sky coordinates.

So, in a few words I had no idea of what I was going to do (still), I realized I didn't have any idea, and the only thing I undestood was how CCD detectors work. I didn't know I had a research adventure awaiting for me.

\section{Objective}\index{Objective}
After I arrived and had my first meeting with Pauline, she explained me a general idea of what she wanted and shared me some more papers (about multi-wavelenght studies), I read the information and came up with the objective.

\begin{itemize}
\item Find out a method to transform data from a high dimensional dataset (FITS cube or any other data arrangement) to a low dimensional understandable information (graphs, clusters).
\end{itemize}

This means that from multiple images with different wavelengths of the same target apply an algorithm to find the hidden patterns that lie hidden between them.

\section{A bit of context}\index{Context}
Ok, here is where I explain from where this is going to start, at that time I just had a microcontrollers and engineering design course my mind was set completelly to find appplicable theories and create uselful things with them, which is the complete opposite of how astronomy works. First, there's no way to test an experiment with galaxies and most of the information is fuzzy and subjective (not all). The process of having an, let's say \emph{astronomy idea} is a result of applying all your physics knowledge and consider the \textbf{cosmological principle},
\begin{quote}
The (testable) assumption that the same physical laws that apply here and now also apply everywhere and at all times, and that there are no special locations or directions in the universe.
\end{quote}

That's how science is made, thinking and testing and thinking again, creating your own scientific method, comming up with hypothesis, learning what might work and what not, using your insticts. 

Well, before comming here I didn't think like that, it was just all about being super productive and thinking about doing robots and all kinds of devices with sensors. I had some experience programming in C/C++, no computer science backgound and I had never had an astronomy course.

This report was written in order to help someone to continue researching about data mining techniques applied in Astronomy, I explain how did I come up with the clustering techniques, my hypothesis, some tests and other ideas I have had, I hope this can help anyone and the research is continued. Anything you may need/questions do not hesitate to contact me, my e-mail address is: \emph{mrs.petzl@gmail.com}, also s part of my own documentation I created a GitHub page where you can download all the codes I programmed and find more information. The link to this page is: \url{https://github.com/LaurethTeX/Clustering}, from the \textsc{readme} file you can acces to all the pages, take your time to surf.
%------------------------------------------------

\subsection{References}\index{References}

Since I found so much good information about pretty much everything I wanted to know about, I will just create a remark and let you know where you can find more specific information about, just like below.

\begin{remark}
For more information about the cosmological principle, review Chapter 1: Why Learn Astronomy?, page 10, from \textbf{21st Century Astronomy}, \textit{Hester | Smith | Blumenthal | Kay | Voss}, Third Edition, 2010.
\end{remark}

%This statement requires citation \cite{book_key}; this one is more specific \cite[122]{article_key}.