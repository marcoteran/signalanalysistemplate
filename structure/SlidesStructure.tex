%----------------------------------------------------------------------------------------
%	Codification text & symbols
%----------------------------------------------------------------------------------------
%\usepackage[utf8]{inputenc} % Required for including letters with accents
%\usepackage[T1]{fontenc} % Use 8-bit encoding that has 256 glyphs
%\usepackage[spanish]{babel}			% Español %\usepackage[spanish]{layout}

%----------------------------------------------------------------------------------------
%	Varios Required Packages
%----------------------------------------------------------------------------------------
\usepackage{comment}			% Agregar comentarios
\usepackage{lipsum}				% Inserts dummy text
%\usepackage{enumitem}			% Customize lists
%\setlist[description]{style=nextline}
\usepackage{blindtext}			% Inserts random paragraph \blindtext
\usepackage{etoolbox}			% Conditional macros
%\setlist{nolistsep}				% Reduce spacing between bullet points and numbered lists
%----------------------------------------------------------------------------------------
%	Colours Packages
%----------------------------------------------------------------------------------------
%\usepackage[table,x11names,dvipsnames,table]{xcolor}	% \usepackage{xcolor}%Specifying colors by name
\usepackage{xcolor}										% Specifying colors by name
%\usepackage[colorlinks=true, urlcolor=ocre]{hyperref}	% Forinserting hyperlinks (colours)
\usepackage{tcolorbox}									% Coloured boxes, for LATEX examples and theorems, etc
\usepackage{color}										% Color packages foreground and back­ground color man­age­ment
%----------------------------------------------------------------------------------------
%	Text and fonts
%----------------------------------------------------------------------------------------
\usepackage{textcomp}					% Text Companion fonts, which provide many text symbols
%\usepackage{titlesec}					% Allows customization of titles
\usepackage{verbatim}					% Verbatim
%\usepackage{microtype}					% Slightly tweak font spacing for aesthetics
\usepackage{avant}						% Use the Avantgarde font for headings
%\RequirePackage{fix-cm}					% permit Computer Modern fonts at arbitrary sizes.
%\usepackage{calc}						% For simpler calculation - used for spacing the index letter headings correctly
\usepackage[footnotesize,hang]{caption} % réduire la taille des légendes des images
%\usepackage[labelsep=endash]{caption}	% Type of separation Caption\usepackage{caption}
\setbeamertemplate{caption}[numbered]
%\usepackage{textcomp}					%
%\usepackage{setspace}					% 
%\usepackage{listings}					% Coding
%\usepackage[framed,numbered,autolinebreaks,useliterate]{./structure/mcode}
%\usepackage{./structure/mcode}			%
\usepackage{times}						% Use the Times font for headings
%\usepackage{lmodern}					%
%\usepackage{marvosym}					%
%\usepackage{gensymb}					%
%\usepackage{mathpazo}					% \textonehalf, \textonequarter, \textthreequarters.
%\usepackage{textcomp}					%
%\usepackage{kpfonts}					% 

%----------------------------------------------------------------------------------------
%	Tables
%----------------------------------------------------------------------------------------
\usepackage{multicol}
%\usepackage{bigstrut}
\usepackage{multirow}       % Allow table cells to span multiple rows
%\usepackage{threeparttable} % tables with footnotes, capions all the same width
%\usepackage{dcolumn}        % decimal-aligned tabular math columns
\usepackage{booktabs}       % Publication-quality tables -nicer horizontal rules in tables
%\usepackage{ltxtable}       % long tabularx
\usepackage{tabularx}
\usepackage{tabulary}
%\usepackage{colortbl}		% colured ‘panels’ behind specified columns in a table.
\usepackage{makecell}		% Celdas con diagonales
\usepackage{longtable}		% Длинные таблицы

%----------------------------------------------------------------------------------------
%	Image packages
%----------------------------------------------------------------------------------------
\usepackage{graphicx}			% Required for including pictures
\usepackage{graphics}			% 
\usepackage{subfig}				% Required for creating figures with multiple parts (subfigures)
\usepackage{tikz}				% Required for drawing custom shapes
%\usetikzlibrary{matrix}			%
%\usetikzlibrary{circuits}		%
%\usetikzlibrary{arrows,shapes}	%
%\usetikzlibrary{chains}			%
%\usetikzlibrary{dsp}			%
%\usetikzlibrary{backgrounds}	
%\usepackage{float}
%\usepackage{subfloat}
%\usepackage{placeins}
%\usepackage[siunitx]{circuitikz}
%\usepackage{schemabloc}
%\usepackage{epstopdf}
%\usepackage{eso-pic}			% Required for specifying an image background in the title page
%----------------------------------------------------------------------------------------
%	Math packages
%----------------------------------------------------------------------------------------
\usepackage{amsmath,amssymb,amsthm}		% For including math equations, theorems, symbols, etc
\usepackage{amsfonts}
%\usepackage{dsfont}
\usepackage{array}
\usepackage{eqnarray}
\usepackage{cases}						% function defined piecewise
\usepackage{mathtools}
\usepackage{mathptmx}					% AdobeTimesRoman as the default text font with mathsymb from Sym­bol/Chancery/Com­puterModernfonts
\usepackage{bm}							% bold math symbols $\alpha \not= \bm{\alpha}$
\usepackage{nicefrac}					%
\usepackage{xfrac}
\usepackage{mathrsfs}