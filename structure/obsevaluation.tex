Se realizará el análisis de disponibilidad de sensores y su relación con la observabilidad si se pierde su información y su efecto en la calidad de la estimación,
para los siguientes distintos siete casos:
\begin{enumerate}
	\item En presencia de solo el sensor GPS y asusencia de los sensores de odometría e inerciales (IMU), $\mH_{\GPS}$
	\item En presencia del sensor odometrico y asusencia de los sensores de GPS e IMU, $\mH_{\odo}$
	\item En presencia de solo la IMU, , $\mH_{\IMU}$
	\item En ausencia del sensor GPS, $\mH_{\IMU+\odo}$
	\item En ausencia de la IMU, $\mH_{\GPS+\odo}$
	\item En ausencia de las mediciones odométricas, $\mH_{\GPS+\IMU}$ 
	\item En presencia de todos los sensores, $\mH$ 
\end{enumerate}

\begin{comment}

Se evaluó la observabilidad para el sistema con función de transición ec.~\ref{straightmodel}.
El Jacobiano de la función se evaluo en un estado aleatorio $\x_k$ diferente de cero.\par
Para evaluar la observabilidad en presencia de solo el sensor de GPS, se resuelve
\begin{equation}
\mathcal{O}_{\GPS}=
\begin{bmatrix} 
H_{\GPS}\\
H_{\GPS}F_{x,k}\\
\vdots\\
H_{\GPS}F_{x,k}^{7-1}
\end{bmatrix}
\end{equation}
donde,
\begin{equation}
\mH_{\GPS} =
\begin{bmatrix} 
1&0&0&0&0&0&0\\ %xgps
0&1&0&0&0&0&0 %ygps
\end{bmatrix}
\end{equation}
El rango de la matriz de observabilidad obtenido $\rank(\mathcal{O}_{\GPS})=5$, menor al número de estados $7$.
Esto implica que utilizando la medida de los sensores GPS, no todos los estados logran ser observables.
Existe inestabilidad en la ecuación de Riccati.\par

Para determinar la observabilidad en presencia de solo el sensor odométrico, se evalúa el rango de la matriz
\begin{equation}
\mathcal{O}_{\odo}=
\begin{bmatrix} 
H_{\odo}\\
H_{\odo}F_{x,k}\\
\vdots\\
H_{\odo}F_{x,k}^{7-1}
\end{bmatrix}
\end{equation}
donde,
\begin{equation}
\mH_{\odo} =
\begin{bmatrix} 
0&0&0&1&0&0&0\\ %dotthetaodo
0&0&0&0&1&0&0\\ %vodo
0&0&0&0&0&1&0 %phiodo
\end{bmatrix}
\end{equation}
El rango $\rank(\mathcal{O}_{\odo})=3$.
No todos los estados son observables y la ecuación de Ricatti presenta inestabilidad.\par

La matriz de observabilidad utilizada para determinar la observabilidad en presencia de solo la unidad de medición inercial se define
\begin{equation}
\mathcal{O}_{\IMU}=
\begin{bmatrix} 
H_{\IMU}\\
H_{\IMU}F_{x,k}\\
\vdots\\
H_{\IMU}F_{x,k}^{7-1}
\end{bmatrix}
\end{equation}
donde,
\begin{equation}
\mH_{\IMU} =
\begin{bmatrix} 
1&0&0&0&0&0&0\\ %ximu
0&1&0&0&0&0&0\\ %yimu
0&0&1&0&0&0&0\\ %thetaimu
0&0&0&0&1&0&0	%vimu
\end{bmatrix}
\end{equation}
El rango utilizando sensores inerciales $\rank(\mathcal{O}_{\IMU})=7$
Es decir, todos los estados son observables.\par

Para determinar la observabilidad utilizando los sensores de la IMU y odometría, se utiliza la siguiente matriz de observabilidad,
\begin{equation}
\mathcal{O}_{\IMU+\odo}=
\begin{bmatrix} 
H_{\IMU+\odo}\\
H_{\IMU+\odo}F_{x,k}\\
\vdots\\
H_{\IMU+\odo}F_{x,k}^{7-1}
\end{bmatrix}
\end{equation}
donde,
\begin{equation}
\mH_{\IMU+\odo} =
\begin{bmatrix} 
1&0&0&0&0&0&0\\ %ximu
0&1&0&0&0&0&0\\ %yimu
0&0&1&0&0&0&0\\ %thetaimu
0&0&0&1&0&0&0\\ %dotthetaodo
0&0&0&0&1&0&0\\ %vodo
0&0&0&0&0&1&0\\ %phiodo
0&0&0&0&1&0&0	%vimu
\end{bmatrix}
\end{equation}
El rango obtenido $\rank(\mathcal{O}_{\IMU+\odo})=5$, lo cual quiere decir que no todos los estados son observables. El filtro de Kalman no se estabiliza.\par

La matriz de observabilidad utilizando los sensores de GPS y odométricos  se define,
\begin{equation}
\mathcal{O}_{\GPS+\odo}=
\begin{bmatrix} 
H_{\GPS+\odo}\\
H_{\GPS+\odo}F_{x,k}\\
\vdots\\
H_{\GPS+\odo}F_{x,k}^{7-1}
\end{bmatrix}
\end{equation}
donde,
\begin{equation}
\mH_{\GPS+\odo} =
\begin{bmatrix} 
1&0&0&0&0&0&0\\ %xgps
0&1&0&0&0&0&0\\ %ygps
0&0&0&1&0&0&0\\ %dotthetaodo
0&0&0&0&1&0&0\\ %vodo
0&0&0&0&0&1&0 %phiodo
\end{bmatrix}
\end{equation}
El rango obtenido es $\rank(\mathcal{O}_{\IMU+\odo})=5$, lo cual quiere decir que no todos los estados son observables y el filtro no se estabiliza.\par

El grado de observabilidad utilizando los sensores de GPS e IMU se define mediante la matriz,
\begin{equation}
\mathcal{O}_{\GPS+\IMU}=
\begin{bmatrix} 
H_{\GPS+\IMU}\\
H_{\GPS+\IMU}F_{x,k}\\
\vdots\\
H_{\GPS+\IMU}F_{x,k}^{7-1}
\end{bmatrix}
\end{equation}
donde,
\begin{equation}
\mH_{\GPS+\IMU} =
\begin{bmatrix} 
1&0&0&0&0&0&0\\ %xgps
0&1&0&0&0&0&0\\ %ygps
1&0&0&0&0&0&0\\ %ximu
0&1&0&0&0&0&0\\ %yimu
0&0&1&0&0&0&0\\ %thetaimu
0&0&0&0&1&0&0	%vimu
\end{bmatrix}
\end{equation}
El rango de la matriz de observabilidad $\rank(\mathcal{O}_{\GPS+\IMU})=7$
Este resultado significa que todos los estados son observables cuando se utilizan simultáneamente sensores de GPS e IMU.\par

Para el caso en presencia de todos los sensores, la matriz de observaciones esta definida por,
\begin{equation}
\mH =
\begin{bmatrix} 
1&0&0&0&0&0&0\\ %xgps
0&1&0&0&0&0&0\\ %ygps
1&0&0&0&0&0&0\\ %ximu
0&1&0&0&0&0&0\\ %yimu
0&0&1&0&0&0&0\\ %thetaimu
0&0&0&1&0&0&0\\ %dotthetaodo
0&0&0&0&1&0&0\\ %vodo
0&0&0&0&0&1&0\\ %phiodo
0&0&0&0&1&0&0	%vimu
\end{bmatrix}
\end{equation}
La matriz de observabilidad utilizando todos los sensores se calcula,
\begin{equation}
\mathcal{O}=
\begin{bmatrix} 
H\\
HF_{x,k}\\
\vdots\\
HF_{x,k}^{7-1}
\end{bmatrix}
\end{equation}
El rango de la matriz de observabilidad $\rank(\mathcal{O})=7$, es decir, el modelo es totalmente observable en el punto $\x_0$.
\end{comment}