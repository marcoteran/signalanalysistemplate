Uno de los aspectos más importantes a tener en cuenta durante el proceso de optimización es la propiedad de observabilidad de el sistema.
La observabilidad es una medición que determina cómo los estados internos pueden ser inferidos a través de las salidas externas.
La observabilidad evalúa el grado de inferencia de los estados del sistema a través de las medidas obtenidas en el vector de observaciones.
Un sistema se define observable si en tiempo finito, durante la transición de los vectores de estado y de control, el estado actual se puede inferir utilizando solamente las medidas del vector de observación.
De acuerdo a la teoría de ingeniería de control \cite{Ogata2010}, un sistema es observable cuando es posible la reconstrucción de su estado inicial, en un intervalo de tiempo, conociendo sus entradas y salidas.
En otras palabras, un sistema observable contiene toda la necesaria información para realizar una estimación, con un error limitado.
Un sistema no es observable cuando los valores de algunos elementos del vector de estado en un momento $k_0$ no puede ser determinado a partir
de las observaciones (salidas) y entradas registradas del sistema.
La observabilidad de un sistema implica un error limitado en la localización.
Los limites del error dependen de la nivel de precisión de los sensores. \par
En un sistema lineal solo un estado podría ser una estimación optima del sistema.
Un sistema lineal no es observable, cuando infinitos estados pueden ser estimaciones optimas
El  proceso de encontrar la observabilidad de sistemas no lineales es mucho más complejo que para procesos lineales.
En un sistema no lineal existe numero considerable de estados que podrían ser estimaciones optimas locales del estado.
En los problemas de localización, las entradas de control, como la velocidad lineal y la angular, son directamente estimadas por los sensores odométricos.
Las salidas son las medidas y observaciones realizadas sobre el sistema \cite{Hermann1977}.\par
Un filtro de Kalman diseñado para un sistema no funcionará con estados no observables.
Un estado es no observable porque no cuenta con la información suficiente de parte de las ecuaciones de observación, en ausencia de esta información la estimación del filtro no convergirá a una solución con sentido.

\subsection{Observabilidad en sistemas lineales}
\label{subsec:linealobs}

Para sistemas lineales de parámetros constantes, la observabilidad esta caracterizada por la \textit{condición de rango}.
Un sistema lineal es observable si y solo si su rango de la matriz
$[H\,HA\, \ldots HA^{n-1} ]$ es igual a la dimensión $n$ de el espacio de estados.

El espacio que se extiende a través de las funciones $Hx,\,HAx,\, \ldots HA^{n-1}x$ es el espacio de observación del sistema.
$\d{\mathcal{O}}$ es el espacio de diferenciales de los elementos de $\mathcal{O}$.
$\d{\mathcal{O}(x)}$ evaluado en $\mathbf{x}$ tiene un rango igual a $n$ para cualquier $x$
entonces se dice que el sistema satisface la condición de observabilidad.

El sistema es observable si la matriz de observabilidad $M_{obs}$ es de rango completo,
donde $\mM_{obs}$ se define
\begin{equation}
	\mM_{obs}=
	\begin{bmatrix} 
		\mH\\
		\mH \mA\\
		\vdots\\
		\mH \mA^i\\
		\vdots\\
		\mH \mA^{n-1}
		\end{bmatrix}
		\label{obsmatrix}
\end{equation}

y se cumple que,

\begin{equation}
\begin{bmatrix} 
\mH\\
\mH \mA\\
\vdots\\
\mH \mA^{n-1}
\end{bmatrix} =
\begin{bmatrix} 
y_0\\
y_1\\
\vdots\\
y_{n-1}
\end{bmatrix}
\end{equation}


Un sistema con un vector de estados $\mathbf{x}$ de dimensión $m$ es observable si la matriz de observabilidad $\M_{obs}$ tiene un rango igual a $m$,
es decir tiene $n$ columnas linalmente independientes \cite{Ogata2010}, \cite{Sinha2007}.

\subsection{Observabilidad en sistemas no lineales}
\label{subsec:nolinealobs}

Para los sistemas no lineales es necesario hacer análisis de observabilidad y controlabilidad.
Los métodos utilizados para determinar la observabilidad en sistemas lineales no aplican en los sistemas no lineales.
En \cite{Hermann1977} se realiza un análisis de observabilidad completo para sistemas no lineales.\par
En el articulo \cite{Martinelli2011} se trata el problema de localización cooperativa de robots móviles.%, equipados con sensores proprioceptivos y extereoceptivos.
Se realiza un análisis de observabilidad teniendo en cuenta las no linealidades del sistema propuesta.
Como solución, se aplica la condición de rango de observabilidad propuesta por Hermann y Krener \cite{Hermann1977}, 
para definir si un sistema definido por $n$ observaciones tiene la propiedad de distinguibilidad local.
%La localización es posible tratarla como un problema de optimización.
%Para garantizar una localización no lineal, se hace muchas veces necesario implementar procesos de de estimación redundantes.\par
%Concepts such as output-to-state stability [20] offer promise for a rigorous mathematical definition of nonlinear observability, but currently no easily implemented tests for such determination exist.
Un método aproximado para determinar la observabilidad en sistemas no lineales es examinado la variación de este en el tiempo de Gramian.
%timevarying Gramian [3].
El espacio de observación es el subespacio vectorial más pequeño de funciones de $\R^n$ cuyos valores pertenecen al espacio de salida.
Estos a su vez contienen $h_1,\,h_2,\ldots,h_p$ y los cuales son aproximadamente es cercano a la derivación de Lie
con respecto a los campos vectoriales del tipo $f_u(x)=f(\mathbf{x},\mathbf{u})$, $\mathbf{u} \in\R^n$ fijos \cite{Martinelli2006}.

%%

\subsection{Aplicación del EKF para determinar la observabilidad en sistemas no lineales}
\label{subsec:ekfnolinealobs}

El método establece un criterio de evaluación del grado de observabilidad para sistemas lineales variantes en el tiempo mediante la aproximación de los sistemas no lineales a sistemas lineales variantes en el tiempo.
Se puede obtener una aproximación de la observabilidad de un sistema no lineal implementando linealizaciones locales de este.
%evaluando el numero condicional de la varianza en el tiempo de Gramian
%Gramians indicate poor observability because different initial conditions can reconstruct the data arbitrarily closely [6].
El EKF linealiza sistemas no lineales en un estado $\x_k$ y luego aplica el filtro de Kalman lineal sobre el modelo linealizado.
Los Jacobianos de la linealización del EKF sustituyen a las matrices $\mA$ y $\mH$ del sistema no lineal.
%luego aplica el filtro de Kalman 
%()el optimo, sin restricciones estimador)
%para obtener las estimaciones del estado
%La aproximación que se realiza: la estadística del proceso son distribuciones normales multivariable
%EKF presented by Stengel [17],
%Un filtro extendido de Kalman es la linealización de un Filtro de Kalman para un sistema no lineal,
Para el sistema planteado en el problema, se asume que $u\equiv0$ es una entrada universal.
El Jacobiano de $\{\xi_1,\,\xi_2,\ldots,\xi_n \} = \{h,\,L_fh,\ldots,L_f^{n-1}h\}$ respecto a $\x$ tiene un rango igual $n$ en el punto $\x_0$.\par
Los criterios de observabilidad definidos para sistemas lineales no aplican para sistemas no lineales.
Es posible realizar sustituciones mediante series de Taylor truncadas que permitan la linealización del sistema.
Una solución consiste en implementar el Jacobiano de la función de transición $f$ y de la función $H$ si fuese necesario necesario en la matriz de observabilidad \ref{obsmatrix}.
El Jacobiano de la función $f$ es evaluado en $x=\hat{x}(k+1|k)$ y reemplaza la matriz de transición $\mA$ \cite{Southall1998}.
\begin{equation}
	\mM_{obs}=
	\begin{bmatrix} 
	\mH\\
	\mH \mF(\x,k)\\
	\vdots\\
	\mH \mF(\x,k)^i\\
	\vdots\\
	\mH \mF(\x,k)^{n-1}
	\end{bmatrix}
\end{equation}
Dentro de la teoría de filtros Kalman, gracias a la ecuación de Riccati (ec.~\ref{riccati}) es posible conocer la evolución de la varianza del estado $P_k$.
Su estabilidad de un filtro de Kalman su puede determinar a partir de los siguientes teoremas:
\begin{itemize}
\item[-] Si $(\mA,\mC)$ es detectable $\forall \mP_0>0$, la solución a la ecuación de Riccati es acotada y generará $\mP_k>0$.
\item[-] Sea $\mQ=\sqrt{\mQ}\sqrt{\mQ}^\intercal>0$ y $\mR>0$, si $(\mA,\mB \sqrt{\mQ})$ son alcanzables y $(\mA,\mC)$ detectable, entonces la solución a la ecuación de Riccati es única con $\mP_k>0$.
\end{itemize}
Donde, $\mA$, $\mB$ y $\mC$ son las matrices del espacio de estados del sistema y $\mQ$ y $\mR$ ruidos de proceso y de medida respectivamente.\par

\begin{equation}
\mP_{k+1}=\mA P_k \mA^\intercal-\mP_k\mC^\intercal\left(\mC \mP_k\mC^\intercal+\mR\right)^{-1}\mC\mP_k\mA^\intercal+\mB\mQ\mB^\intercal
\label{riccati}
\end{equation}

\begin{comment}

La ecuación de movimiento del vehículo plano de 4 ruedas propuesta por el articulo \cite{Toledo-Moreo2007} se define mediante la ecuación ~\ref{straightmodel}.
El Jacobiano de esta función de transición está definido por,
\begin{equation}
\begin{array}{rl}
\mF(\x,k)&=\left. \pdv{f(\mathbf{x},k)}{\mathbf{x}}\right\rvert_{\mathbf{x}=\mathbf{x}_k}\\
&=
{ 
\begin{bmatrix}
	1& 0&
	\mF_{1,3}(\x,k)
	& \mF_{1,4}(\x,k)
	& \mF_{1,5}(\x,k)
	& \mF_{1,6}(\x,k)
	& \mF_{1,7}(\x,k)\\
	
	
	0& 1&  \mF_{2,3}(\x,k)
	&  \mF_{2,4}(\x,k)
	& \mF_{2,5}(\x,k)  
	& \mF_{2,6}(\x,k)
	& \mF_{2,7}(\x,k)\\
	
	
	0 & 0 & 1 & 0 & 0 & \DeltaT & 0\\
	0 & 0 & 0 & 1 & 0 & 0 & 0\\
	0 & 0 & 0 & 0 & 1 & 0 & 0\\
	0 & 0 & 0 & 0 & 0 & 1 & 0\\
	0 & 0 & 0 & 0 & 0 & 0 & 1
	
	\end{bmatrix}}
\end{array}
\end{equation}

Donde,
\begin{equation*}
\begin{array}{rl}
	\mF_{1,3}(\x,k)&=- \DeltaT v_k \sin(\phi_k + \theta_k + s_k) - \frac{1}{2}\DeltaT^2\dot{\theta}_k v_k\cos(\phi_k + \theta_k + s_k)\\
	\mF_{1,4}(\x,k)&=-\frac{1}{2}\DeltaT^2 v_k\sin(\phi_k + \theta_k + s_k)\\
	\mF_{1,5}(\x,k)&= \DeltaT\cos(\phi_k + \theta_k + s_k) - \frac{1}{2}\DeltaT^2\dot{\theta}_k\sin(\phi_k + \theta_k + s_k)\\
	\mF_{1,6}(\x,k)&=- \DeltaT v_k \sin(\phi_k + \theta_k + s_k)  - \frac{1}{2}\DeltaT^2\dot{\theta}_k v_k\cos(\phi_k + \theta_k + s_k)\\
	\mF_{1,7}(\x,k)&= - \DeltaT v_k \sin(\phi_k + \theta_k + s_k) - \frac{1}{2}\DeltaT^2\dot{\theta}_k v_k\cos(\phi_k + \theta_k + s_k)\\
	\mF_{2,3}(\x,k)&= \DeltaT v_k \cos(\phi_k + \theta_k + s_k) - \frac{1}{2}\DeltaT^2\dot{\theta}_k v_k\sin(\phi_k + \theta_k + s_k)\\
	\mF_{2,4}(\x,k)&=\frac{1}{2}\DeltaT^2 v_k\cos(\phi_k + \theta_k + s_k)\\
	\mF_{2,5}(\x,k)&=\frac{1}{2}\DeltaT^2 \dot{\theta}_k \cos(\phi_k + \theta_k + s_k)  + \DeltaT\sin(\phi_k + \theta_k + s_k)\\
	\mF_{2,6}(\x,k)&=\DeltaT v_k \cos(\phi_k + \theta_k + s_k) - \frac{1}{2}\DeltaT^2\dot{\theta}_k v_k\sin(\phi_k + \theta_k + s_k)\\
	\mF_{2,7}(\x,k)&=  \DeltaT v_k \cos(\phi_k + \theta_k + s_k) - \frac{1}{2}\DeltaT^2\dot{\theta}_k v_k\sin(\phi_k + \theta_k + s_k)
\end{array}
\end{equation*}

\end{comment}

Vale la pena destacar, que debido a que el EKF linealiza el modelo no lineal y luego evalúa en el estado del sistema estimado,
no es posible determinar una sola matriz de observabilidad que no dependa del estado en cierto momento.