%----------------------------------------------------------------------------------------
%	THEOREM STYLES
%----------------------------------------------------------------------------------------
\newcommand{\intoo}[2]{\mathopen{]}#1\,;#2\mathclose{[}}
\newcommand{\ud}{\mathop{\mathrm{{}d}}\mathopen{}}
\newcommand{\intff}[2]{\mathopen{[}#1\,;#2\mathclose{]}}
\newtheorem{notation}{Notación}[chapter]

% Boxed/framed environments
\newtheoremstyle{ocrenumbox}% % Theorem style name
{0pt}% Space above
{0pt}% Space below
{\normalfont}% % Body font
{}% Indent amount
{\small\bf\sffamily\color{ocre}}% % Theorem head font
{\;}% Punctuation after theorem head
{0.25em}% Space after theorem head
{\small\sffamily\color{ocre}\thmname{#1}\nobreakspace\thmnumber{\@ifnotempty{#1}{}\@upn{#2}}% Theorem text (e.g. Theorem 2.1)
	\thmnote{\nobreakspace\the\thm@notefont\sffamily\bfseries\color{black}---\nobreakspace#3.}} % Optional theorem note
\renewcommand{\qedsymbol}{$\blacksquare$}% Optional qed square

\newtheoremstyle{blacknumex}% Theorem style name
{5pt}% Space above
{5pt}% Space below
{\normalfont}% Body font
{} % Indent amount
{\small\bf\sffamily}% Theorem head font
{\;}% Punctuation after theorem head
{0.25em}% Space after theorem head
{\small\sffamily{\tiny\ensuremath{\blacksquare}}\nobreakspace\thmname{#1}\nobreakspace\thmnumber{\@ifnotempty{#1}{}\@upn{#2}}% Theorem text (e.g. Theorem 2.1)
	\thmnote{\nobreakspace\the\thm@notefont\sffamily\bfseries---\nobreakspace#3.}}% Optional theorem note

\newtheoremstyle{blacknumbox} % Theorem style name
{0pt}% Space above
{0pt}% Space below
{\normalfont}% Body font
{}% Indent amount
{\small\bf\sffamily}% Theorem head font
{\;}% Punctuation after theorem head
{0.25em}% Space after theorem head
{\small\sffamily\thmname{#1}\nobreakspace\thmnumber{\@ifnotempty{#1}{}\@upn{#2}}% Theorem text (e.g. Theorem 2.1)
	\thmnote{\nobreakspace\the\thm@notefont\sffamily\bfseries---\nobreakspace#3.}}% Optional theorem note

% Non-boxed/non-framed environments
\newtheoremstyle{ocrenum}% % Theorem style name
{5pt}% Space above
{5pt}% Space below
{\normalfont}% % Body font
{}% Indent amount
{\small\bf\sffamily\color{ocre}}% % Theorem head font
{\;}% Punctuation after theorem head
{0.25em}% Space after theorem head
{\small\sffamily\color{ocre}\thmname{#1}\nobreakspace\thmnumber{\@ifnotempty{#1}{}\@upn{#2}}% Theorem text (e.g. Theorem 2.1)
	\thmnote{\nobreakspace\the\thm@notefont\sffamily\bfseries\color{black}---\nobreakspace#3.}} % Optional theorem note
\renewcommand{\qedsymbol}{$\blacksquare$}% Optional qed square
\makeatother

% url environment
\DeclareUrlCommand{\url}{%
	\def\UrlFont{\color{ocre}\normalfont}%      Adding a little color 
	%\def\UrlLeft##1\UrlRight{\underline{##1}}%  Underlining the url
}


% Non-boxed/non-framed environments
\newtheoremstyle{rednum}% % Theorem style name
{5pt}% Space above
{5pt}% Space below
{\normalfont}% % Body font
{}% Indent amount
{\small\bf\sffamily\color{ocre}}% % Theorem head font
{\;}% Punctuation after theorem head
{0.25em}% Space after theorem head
{\small\sffamily\color{ocre}\thmname{#1}\nobreakspace\thmnumber{\@ifnotempty{#1}{}\@upn{#2}}% Theorem text (e.g. Theorem 2.1)
	\thmnote{\nobreakspace\the\thm@notefont\sffamily\bfseries\color{black}---\nobreakspace#3.}} % Optional theorem note
\renewcommand{\qedsymbol}{$\blacksquare$}% Optional qed square
\makeatother

% Defines the theorem text style for each type of theorem to one of the three styles above
\newcounter{dummy} 
\numberwithin{dummy}{section}

\theoremstyle{ocrenumbox}
\newtheorem{theoremeT}[dummy]{Teorema}
\newtheorem{exampleT}{Ejemplo}[chapter]


\theoremstyle{blacknumex}
\theoremstyle{blacknumbox}
\newtheorem{vocabulary}{Vocabulary}[chapter]
\newtheorem{definitionT}{Definición}[section]
\newtheorem{corollaryT}[dummy]{Corolario}
\newtheorem{problem}{Problema}[chapter]
\newtheorem{exerciseT}{Ejercicio}[chapter]

\theoremstyle{ocrenum}
\newtheorem{proposition}[dummy]{Proposition}


%----------------------------------------------------------------------------------------
%	DEFINITION OF COLORED BOXES
%----------------------------------------------------------------------------------------

\RequirePackage[framemethod=default]{mdframed} % Required for creating the theorem, definition, exercise and corollary boxes

% Theorem box
\newmdenv[skipabove=7pt,
skipbelow=7pt,
backgroundcolor=black!5,
linecolor=ocre,
innerleftmargin=5pt,
innerrightmargin=5pt,
innertopmargin=5pt,
leftmargin=0cm,
rightmargin=0cm,
innerbottommargin=5pt]{tBox}

% Exercise box	  
\newmdenv[skipabove=7pt,
skipbelow=7pt,
rightline=false,
leftline=true,
topline=false,
bottomline=false,
backgroundcolor=darkred!10,
linecolor=darkred,
innerleftmargin=5pt,
innerrightmargin=5pt,
innertopmargin=5pt,
innerbottommargin=5pt,
leftmargin=0cm,
rightmargin=0cm,
linewidth=4pt]{eBox}	

% Example box
\newmdenv[skipabove=7pt,
skipbelow=7pt,
rightline=false,
leftline=true,
topline=false,
bottomline=false,
linecolor=ocre,
backgroundcolor=ocre!5,
innerleftmargin=5pt,
innerrightmargin=5pt,
innertopmargin=5pt,
leftmargin=0cm,
rightmargin=0cm,
linewidth=4pt,
innerbottommargin=5pt]{ejBox}

% Definition box
\newmdenv[skipabove=7pt,
skipbelow=7pt,
rightline=false,
leftline=true,
topline=false,
bottomline=false,
linecolor=ocre,
innerleftmargin=5pt,
innerrightmargin=5pt,
innertopmargin=0pt,
leftmargin=0cm,
rightmargin=0cm,
linewidth=4pt,
innerbottommargin=0pt]{dBox}	

% Corollary box
\newmdenv[skipabove=7pt,
skipbelow=7pt,
rightline=false,
leftline=true,
topline=false,
bottomline=false,
linecolor=gray,
backgroundcolor=black!5,
innerleftmargin=5pt,
innerrightmargin=5pt,
innertopmargin=5pt,
leftmargin=0cm,
rightmargin=0cm,
linewidth=4pt,
innerbottommargin=5pt]{cBox}

% Creates an environment for each type of theorem and assigns it a theorem text style from the "Theorem Styles" section above and a colored box from above
\newenvironment{theorem}{\begin{tBox}\begin{theoremeT}}{\end{theoremeT}\end{tBox}}
\newenvironment{exercise}{\begin{eBox}\begin{exerciseT}}{\hfill{\color{ocre}\tiny\ensuremath{\blacksquare}}\end{exerciseT}\end{eBox}}				  

\newenvironment{example}{\begin{ejBox}\begin{exampleT}}{\hfill{\color{ocre}\tiny\ensuremath{\blacksquare}}\end{exampleT}\end{ejBox}}		

\newenvironment{definition}{\begin{dBox}\begin{definitionT}}{\end{definitionT}\end{dBox}}	
\newenvironment{corollary}{\begin{cBox}\begin{corollaryT}}{\end{corollaryT}\end{cBox}}	

%----------------------------------------------------------------------------------------
%	REMARK ENVIRONMENT
%----------------------------------------------------------------------------------------
\input{./structure/remarkenv}
%----------------------------------------------------------------------------------------
%	SECTION NUMBERING IN THE MARGIN
%----------------------------------------------------------------------------------------
\input{./structure/sectionnumbering}
%----------------------------------------------------------------------------------------
%	PART HEADINGS
%----------------------------------------------------------------------------------------
\input{./structure/partheading}
%----------------------------------------------------------------------------------------
%	CHAPTER HEADINGS
%----------------------------------------------------------------------------------------
\newcommand{\thechapterimage}{}%
\newcommand{\chapterimage}[1]{\renewcommand{\thechapterimage}{#1}}%
\def\@makechapterhead#1{%
{\parindent \z@ \raggedright \normalfont
\ifnum \c@secnumdepth >\m@ne
\if@mainmatter
\begin{tikzpicture}[remember picture,overlay]
\node at (current page.north west)
{\begin{tikzpicture}[remember picture,overlay]
\node[anchor=north west,inner sep=0pt] at (0,0) {\includegraphics[width=\paperwidth]{\thechapterimage}};
\draw[anchor=west] (\Gm@lmargin,-9cm) node [line width=2pt,rounded corners=15pt,draw=ocre,fill=white,fill opacity=0.5,inner sep=15pt]{\strut\makebox[22cm]{}};
\draw[anchor=west] (\Gm@lmargin+.3cm,-9cm) node {\huge\sffamily\bfseries\color{black}\thechapter. #1\strut};
\end{tikzpicture}};
\end{tikzpicture}
\else
\begin{tikzpicture}[remember picture,overlay]
\node at (current page.north west)
{\begin{tikzpicture}[remember picture,overlay]
\node[anchor=north west,inner sep=0pt] at (0,0) {\includegraphics[width=\paperwidth]{\thechapterimage}};
\draw[anchor=west] (\Gm@lmargin,-9cm) node [line width=2pt,rounded corners=15pt,draw=ocre,fill=white,fill opacity=0.5,inner sep=15pt]{\strut\makebox[22cm]{}};
\draw[anchor=west] (\Gm@lmargin+.3cm,-9cm) node {\huge\sffamily\bfseries\color{black}#1\strut};
\end{tikzpicture}};
\end{tikzpicture}
\fi\fi\par\vspace*{270\p@}}}

%-------------------------------------------

\def\@makeschapterhead#1{%
\begin{tikzpicture}[remember picture,overlay]
\node at (current page.north west)
{\begin{tikzpicture}[remember picture,overlay]
\node[anchor=north west,inner sep=0pt] at (0,0) {\includegraphics[width=\paperwidth]{\thechapterimage}};
\draw[anchor=west] (\Gm@lmargin,-9cm) node [line width=2pt,rounded corners=15pt,draw=ocre,fill=white,fill opacity=0.5,inner sep=15pt]{\strut\makebox[22cm]{}};
\draw[anchor=west] (\Gm@lmargin+.3cm,-9cm) node {\huge\sffamily\bfseries\color{black}#1\strut};
\end{tikzpicture}};
\end{tikzpicture}
\par\vspace*{270\p@}}
\makeatother
%----------------------------------------------------------------------------------------
%	PAGE HEADINGS
%----------------------------------------------------------------------------------------
\input{./structure/pageheaders}
%----------------------------------------------------------------------------------------
%----------------------------------------------------------------------------------------
%	CONTENT TABLES
%----------------------------------------------------------------------------------------
\input{./structure/contenttables}
%----------------------------------------------------------------------------------------
%	HYPERLINKS IN THE DOCUMENTS
%----------------------------------------------------------------------------------------
\usepackage{hyperref}
\hypersetup{hidelinks,backref=true,pagebackref=true,hyperindex=true,colorlinks=false,breaklinks=true,urlcolor= ocre,bookmarks=true,bookmarksopen=false,pdftitle={Title},pdfauthor={Author}}
\usepackage{bookmark}
\bookmarksetup{
open,
numbered,
addtohook={%
\ifnum\bookmarkget{level}=0 % chapter
\bookmarksetup{bold}%
\fi
\ifnum\bookmarkget{level}=-1 % part
\bookmarksetup{color=ocre,bold}%
\fi
}
}

\newenvironment{bookboxed}[1]
{\begin{center}
		#1\\[1ex]
		\begin{tabular}{|p{0.9\textwidth}|}
			\hline\\
		}
		{ 
			\\\\\hline
		\end{tabular} 
	\end{center}
}

%\newenvironment{chapterquote}[2][Anonimo]{\noindent Hello, #1}{\noindent Bye now! \#2\\

\DeclareDocumentEnvironment{chapterquote}{O {cake} m }
{ 
	\begin{mdframed}[backgroundcolor=gray!1,fontcolor=black!75, linewidth=0pt]\itshape
	\begin{flushright}
}
{
\end{flushright}
\end{mdframed}
\vspace{-22pt}
\begin{flushright}{\bfseries #1}, {\small #2} \end{flushright}
}

\newenvironment{bproof}[1][\proofname]{%
	\proof[\ttfamily \scshape \large #1]%
}{\endproof}

\makeatletter
\renewenvironment{proof}[1][\proofname]{\par
	\pushQED{\qed}%
	\normalfont \topsep6\p@\@plus6\p@\relax
	\trivlist
	\item[%
	\hskip\labelsep
	%\normalfont\bfseries % was \itshape
	\ttfamily \scshape \large
	#1%
	\@addpunct{.}% remove this if you don't want punctuation
	]\ignorespaces
}{%
	\popQED\endtrivlist\@endpefalse
}
\let\qed\relax % avoid a warning
\DeclareRobustCommand{\qed}{%
	\ifmmode \mathqed
	\else
	\leavevmode\unskip\penalty\@M\hbox{}\nobreak\hspace{.5em minus .1em}% was \hfill
	\hfill \hbox{\qedsymbol}%
	\fi
}
\makeatother

\let\oldproof\proof
\renewcommand{\proof}{\color{black!75}\oldproof}


%Numbered environment
\newcounter{example}[section]
\newenvironment{bexample}[1][]{\refstepcounter{example}\par\medskip
	\noindent \textbf{Example~\theexample. #1} \rmfamily}{\medskip}

%Numbered environment defined with Newtheorem

\newtheorem{SampleEnv}{Sample Environment}[section]



\newtheorem*{bhistoricenv}{{\footnotesize Perpectiva histórica} \vspace{2pt} \\}
\DeclareDocumentEnvironment{bhistory}{O {\unskip} m }
{

\renewcommand\windowpagestuff{\centering\includegraphics[width=2.4cm]{#2}}
\opencutleft

\begin{mdframed}[backgroundcolor=gray!3,fontcolor=black!75,linewidth=0pt]
\vspace{-10pt}
\begin{bhistoricenv} 
\par 
\vspace{15pt}
\begin{cutout}{3}{4pt}{\dimexpr\linewidth-3.8cm\relax}{9}

\noindent \textbf{#1}
}%
{

\end{cutout}
\end{bhistoricenv}\end{mdframed}
}



	
\newtheorem{thm}{Imagen en texto}

\newenvironment{hello}[1][world]{\noindent Hello #1, }{Bye now!\\}

\DeclareDocumentEnvironment{envb}{O {cake} m }
{
	\begin{wrapfigure}{r}{0.4\textwidth}
		\vspace{-20pt}
		\begin{center}
			\includegraphics[width=0.38\textwidth]{#2}
		\end{center}
		\vspace{-20pt}
		\caption{A gull}
		\vspace{-10pt}
		%\label{fig:gauss}
	\end{wrapfigure}
	\textbf{I like #1 and .}
}
{
Hola, FIN!
}


\newcommand{\bluebox}[1]{%
	\setbox0=\hbox{#1}%
	\setlength{\@tempdima}{\dimexpr\wd0+13pt}%
	\begin{tcolorbox}[colback=ocre!5!white,colframe=ocre!75!black,boxrule=0.5pt,arc=4pt,
		left=6pt,right=6pt,top=6pt,bottom=6pt,boxsep=0pt,width=\@tempdima]
		#1
	\end{tcolorbox}
}

